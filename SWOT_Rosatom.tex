%% Необхоимый минимум
\documentclass[a4paper,12pt]{article}

%%% Работа с русским языком
\usepackage{cmap}					% поиск в PDF
\usepackage{mathtext} 				% русские буквы в формулах
\usepackage[T2A]{fontenc}			% кодировка
\usepackage[utf8]{inputenc}			% кодировка исходного текста
\usepackage[english,russian]{babel}	% локализация и переносы

%%% Дополнительная работа с математикой
\usepackage{amsfonts,amssymb,amsthm,mathtools} % AMS
\usepackage{amsmath}
\usepackage{icomma} % "Умная" запятая: $0,2$ --- число, $0, 2$ --- перечисление

%% Номера формул
\mathtoolsset{showonlyrefs=true} % Показывать номера только у тех формул, на которые есть \eqref{} в тексте.

%% Шрифты
\usepackage{euscript}	 % Шрифт Евклид
\usepackage{mathrsfs} % Красивый матшрифт

%% Свои команды
\DeclareMathOperator{\sgn}{\mathop{sgn}}

%% Перенос знаков в формулах (по Львовскому)
\newcommand*{\hm}[1]{#1\nobreak\discretionary{}
	{\hbox{$\mathsurround=0pt #1$}}{}}

\usepackage{gensymb}
%\usepackage{indentfirst}
\usepackage{lscape}

\usepackage{caption}
\captionsetup{labelsep=period}
\captionsetup{justification=centerlast}
\usepackage{subcaption}
\renewcommand{\thesubfigure}{\asbuk{subfigure}}

\usepackage{microtype}


%%% Параметры страницы: поля, колонтитулы, интерлиньяж
\usepackage{extsizes} % Возможность сделать 14-й шрифт
\usepackage{geometry} % Простой способ задавать поля
\geometry{top=12mm}
\geometry{bottom=12mm}
\geometry{left=12mm}
\geometry{right=12mm}

\usepackage{fancyhdr} % Колонтитулы
%\pagestyle{fancy}
\renewcommand{\headrulewidth}{0mm}  % Толщина линейки, отчеркивающей верхний колонтитул
%\rfoot{Нижний правый}
%\rhead{Верхний правый}
%\chead{Верхний в центре}
%\lhead{Верхний левый}
% \cfoot{Нижний в центре} % По умолчанию здесь номер страницы

\usepackage{setspace} % Интерлиньяж
%\onehalfspacing % Интерлиньяж 1.5
%\doublespacing % Интерлиньяж 2
%\singlespacing % Интерлиньяж 1

%%% Работа с таблицами
\usepackage{array,tabularx,tabulary,booktabs} % Дополнительная работа с таблицами
\usepackage{longtable}  % Длинные таблицы
\usepackage{multirow} % Слияние строк в таблице

%%% Работа с картинками
\usepackage{graphicx}  % Для вставки рисунков
\graphicspath{{images/}{images2/}}  % папки с картинками
\setlength\fboxsep{3pt} % Отступ рамки \fbox{} от рисунка
\setlength\fboxrule{1pt} % Толщина линий рамки \fbox{}
\usepackage{wrapfig} % Обтекание рисунков и таблиц текстом

\usepackage{lastpage} % Узнать, сколько всего страниц в документе.

\usepackage{soul} % Модификаторы начертания
\usepackage{soulutf8} % Модификаторы начертания

%\renewcommand{\familydefault}{\sfdefault} % Начертание шрифта

\usepackage{multicol} % Несколько колонок

\begin{document}

\emph{Обедненный уран} "--- уран, изотопный состав которого отличается от природного урана уменьшенной долей изотопа урана-235. В более узком смысле под обедненным ураном понимают уран, в изотопном составе которого доля поддерживающих цепную реакцию изотопов урана (U-235 и U-233) не превышает 0.3 \%.

\paragraph{Основной источник обедненного урана} "--- это заводы обогащения урана, где обедненный уран является отходом производства. также источником обедненного урана могут быть ядерные реакторы, в топливе которых происходит выгорание U-235 до уровней ниже природного, а также заводы по переработке ОЯТ.

\paragraph{Особенности обедненного урана}

\begin{itemize}
	\item Высокая токсичность. В России все химические соединения урана относятся к классу опасности 1 (чрезвычайно опасные вещества). Максимальная разовая концентрация в воздухе рабочей зоны составляет $75~мкг/м^3$ для нерастворимых и $15~мкг/м^3$ для растворимых соединений.
	\item Высокая химическая активность: высокая скорость коррозии, пирофорность "--- способность к самовоспламенению на открытом воздухе в мелкораздробленном состоянии.
	\item Альфа-активность при внутреннем облучении.
	\item Высокая плотность ($19.05~кг/м^3$). 
	\item Невысокая активность (по сравнению с природным ураном "--- 25 против 15 Бк/мг.)
\end{itemize}

\section{Гражданское применение}
В виду своих особенностей обедненный уран в гражданской сфере может быть полезен только в роли компактного балласта или балансировочного груза (в данный момент подобное использование прекращено) или в качестве защиты от гамма-излучения.

В виду своей высокой химической активности обедненный уран не находит большого применения в гражданской области, однако все его недостатки превращаются в достоинства, когда речь заходит от военном применении.

\section{Военное применение}
В военной промышленности обедненный уран применяется в качестве брони тяжело бронированной техники (защищающей одновременно от ионизирующего излучения и от механического пробития), либо в качестве бронебойного сердечника снарядов.

Применение обедненного урана позволяет сильно увеличить заброневой эффект бронебойных подкалиберных снарядов. Эффект обычных БПС сводится к прямому поражению сердечником снаряда, а также образованными при пробитии осколками брони. Урановый сердечник БПС способен нанести гораздо больший разрушительный эффект благодаря своей пирофорности.


\end{document}