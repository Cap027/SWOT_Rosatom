\include{preamble_base.tex}

\begin{document}

\emph{Обедненный уран} "--- уран, изотопный состав которого отличается от природного урана уменьшенной долей изотопа урана-235. В более узком смысле под обедненным ураном понимают уран, в изотопном составе которого доля поддерживающих цепную реакцию изотопов урана (U-235 и U-233) не превышает 0.3 \%.

\paragraph{Основной источник обедненного урана} "--- это заводы обогащения урана, где обедненный уран является отходом производства. также источником обедненного урана могут быть ядерные реакторы, в топливе которых происходит выгорание U-235 до уровней ниже природного, а также заводы по переработке ОЯТ.

\paragraph{Особенности обедненного урана}

\begin{itemize}
	\item Высокая токсичность. В России все химические соединения урана относятся к классу опасности 1 (чрезвычайно опасные вещества). Максимальная разовая концентрация в воздухе рабочей зоны составляет $75~мкг/м^3$ для нерастворимых и $15~мкг/м^3$ для растворимых соединений.
	\item Высокая химическая активность: высокая скорость коррозии, пирофорность "--- способность к самовоспламенению на открытом воздухе в мелкораздробленном состоянии.
	\item Альфа-активность при внутреннем облучении.
	\item Высокая плотность ($19.05~кг/м^3$). 
	\item Невысокая активность (по сравнению с природным ураном "--- 25 против 15 Бк/мг.)
\end{itemize}

\section{Гражданское применение}
В виду своих особенностей обедненный уран в гражданской сфере может быть полезен только в роли компактного балласта или балансировочного груза (в данный момент подобное использование прекращено) или в качестве защиты от гамма-излучения.

В виду своей высокой химической активности обедненный уран не находит большого применения в гражданской области, однако все его недостатки превращаются в достоинства, когда речь заходит от военном применении.

\section{Военное применение}
В военной промышленности обедненный уран применяется в качестве брони тяжело бронированной техники (защищающей одновременно от ионизирующего излучения и от механического пробития), либо в качестве бронебойного сердечника снарядов.

Применение обедненного урана позволяет сильно увеличить заброневой эффект бронебойных подкалиберных снарядов. Эффект обычных БПС сводится к прямому поражению сердечником снаряда, а также образованными при пробитии осколками брони. Урановый сердечник БПС способен нанести гораздо больший разрушительный эффект благодаря своей пирофорности.


\end{document}