\include{preamble_base.tex}

\begin{document}

Энергия атомного ядра на сегодняшний день "--- один из самых чистых, стабильных и мощных источников энергии. Однако на заря развития данной отрасли было допущено множество ошибок, которые привели к ужасным последствиям для человечества, что вселило в головы многих людей страх перед атомной энергетикой. В связи с этим долгое время атомная отрасль пребывает в упадке: новые проекты не находят поддержки, а многие страны Европы предпочитают уходить от выработки энергии на АЭС.

Несмотря на общий негативный фон, сложившийся вокруг атомной отрасли, в мире существует компания "--- <<Росатом>>, которая не только продолжает поддерживать и модернизировать существующие объекты, но и возводить новые, в том числе и за рубежом. Естественно, что управление компанией в столь сложной сфере сопряжено со множеством трудностей (технологических, экономических и политических), поэтому для формирования эффективной корпоративной стратегии управления необходимо применять различные методы анализа внешней и внутренней сред. Для решения данной задачи обычно используется так называемый SWOT-анализ.

\section{SWOT-анализ ГК <<Росатом>>}

\subsection{Сильные стороны (Strengths)}
\begin{enumerate}
\item \emph{Технологическое превосходство}: Росатом является одним из мировых лидеров в области атомной энергетики и обладает передовыми технологиями в этой области. Так, например, России принадлежит технологический приоритет в разработке и эксплуатации реакторов на быстрых нейтронах (в данный момент эксплуатируются БН-600 и БН-800), а также активно развивается технология замкнутого топливного цикла.

\item \emph{Вертикальная интеграция}: организационная структура Росатома поделена на т.н. дивизионы, которые в совокупности контролируют все направления деятельности корпорации. Так, например, предприятия <<горно-рудного>> дивизиона занимаются добычей урано-содержащих руд, <<топливный>> дивизион занимается обогащением урана и производством ядерного топлива, <<машиностроительный>> "--- разрабатывает и производит оборудование ядерного острова и машинного зала для атомных станций, <<инжиниринговый>> "--- реализует проекты по сооружению АЭС, <<электроэнергетический>> "--- занимается вводом в эксплуатацию, эксплуатацией и обслуживанием энергоблоков.

\item \emph{Международное присутствие}: Росатом активно работает на международном рынке, имеет контракты с различными странами и участвует в строительстве атомных электростанций за рубежом. В частности, в этом году был введен в эксплуатацию второй энергоблок на белорусской АЭС, ведется строительство энергоблоков в Бангладеше (Руппур), Турции (Аккую), Египте (Эль-Дабаа), готовится строительство двух энергоблоков в Венгрии (Пакш).

\item \emph{Научно-исследовательская база}: Росатом имеет сильную научно-исследовательскую базу, которая позволяет разрабатывать новые технологии и инновации в области атомной энергетики, а также в областях термоядерного синтеза и ядерной физики.
\end{enumerate}
Weaknesses (Слабые стороны):
1. Риски безопасности: Атомная энергетика связана с определенными рисками, включая возможность аварий и утечек радиации. Это может негативно сказаться на репутации Росатома.
2. Зависимость от государственного финансирования: Росатом является государственной корпорацией и зависит от финансирования со стороны правительства. Это может ограничивать его финансовую независимость и гибкость.
3. Ограниченное разнообразие продуктов: Основной продукцией Росатома является атомная энергия, что ограничивает его возможности в других отраслях энергетики.

Opportunities (Возможности):
1. Рост мирового спроса на энергию: В условиях изменения климата и стремления к снижению выбросов парниковых газов, атомная энергетика может стать более привлекательной альтернативой для многих стран.
2. Развитие новых технологий: Росатом может продолжать разрабатывать и внедрять новые технологии в области атомной энергетики, такие как улучшенные реакторы и возобновляемое ядерное топливо.
3. Международное сотрудничество: Росатом может расширить свое международное сотрудничество и участвовать в проектах с другими странами, что позволит ему получить доступ к новым рынкам и технологиям.

Threats (Угрозы):
1. Конкуренция от возобновляемых источников энергии: Развитие возобновляемых источников энергии, таких как солнечная и ветровая энергия, может представлять угрозу для атомной энергетики и Росатома.
2. Политические риски: Атомная энергетика является политически чувствительной отраслью, и изменения в политической ситуации могут повлиять на деятельность Росатома.
3. Регулятивные ограничения: Росатом может столкнуться с регулятивными ограничениями и требованиями в различных странах, что может затруднить его международную деятельность.

Это лишь некоторые примеры сильных и слабых сторон, возможностей и угроз, и SWOT-анализ Росатома может быть более подробным и уникальным в зависимости от конкретных условий и контекста.

\end{document}